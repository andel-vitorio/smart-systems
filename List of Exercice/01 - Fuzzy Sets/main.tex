\documentclass{article} 

\usepackage{enumitem}
\usepackage{amsmath}
\usepackage{booktabs}
\usepackage{bm}
\usepackage{float}
\usepackage[margin=2cm]{geometry}

\title{Exercise Problems: Fuzzy Sets} 
\author{Andevaldo da Encarnação Vitório} 
\date{\today}

\begin{document}

\maketitle

\begin{enumerate}
  \item Consider two fuzzy sets, one representing a scooter and other van.

        \begin{gather*}
          \mathrm{Scooter} = \left\{
          \frac{0.6}{\mathrm{van}} +
          \frac{0.3}{\mathrm{motor cycle}} +
          \frac{0.8}{\mathrm{boat}} +
          \frac{0.9}{\mathrm{scooter}} +
          \frac{0.1}{\mathrm{house}}
          \right\}, \\[12pt]
          \mathrm{Van} = \left\{
          \frac{1.0}{\mathrm{van}} +
          \frac{0.2}{\mathrm{motor cycle}} +
          \frac{0.5}{\mathrm{boat}} +
          \frac{0.3}{\mathrm{scooter}} +
          \frac{0.2}{\mathrm{house}}
          \right\}
        \end{gather*}

        Find the following:

        \begin{enumerate}
          \item $\mathrm{Scooter} \cup \mathrm{Van} = \boldsymbol{
                    \left\{
                    \frac{1.0}{\mathrm{van}} +
                    \frac{0.3}{\mathrm{motor cycle}} +
                    \frac{0.8}{\mathrm{boat}} +
                    \frac{0.9}{\mathrm{scooter}} +
                    \frac{0.2}{\mathrm{house}}
                    \right\}}$
                \vspace{8pt}

          \item $\mathrm{Scooter} \setminus \mathrm{Van} =
                  \mathrm{Scooter} \cap \overline{\mathrm{Van}} =
                  \boldsymbol{\left\{
                    \frac{0.0}{\mathrm{van}} +
                    \frac{0.3}{\mathrm{motor cycle}} +
                    \frac{0.5}{\mathrm{boat}} +
                    \frac{0.7}{\mathrm{scooter}} +
                    \frac{0.1}{\mathrm{house}}
                    \right\}}$
                \vspace{8pt}

          \item $\mathrm{Scooter} \cap \overline{\mathrm{Van}}  =
                  \boldsymbol{\left\{
                    \frac{0.0}{\mathrm{van}} +
                    \frac{0.3}{\mathrm{motor cycle}} +
                    \frac{0.5}{\mathrm{boat}} +
                    \frac{0.7}{\mathrm{scooter}} +
                    \frac{0.1}{\mathrm{house}}
                    \right\}}$
                \vspace{8pt}

          \item $\overline{\mathrm{Scooter} \cup \mathrm{Scooter}} = \overline{\mathrm{Scooter}}
                  = \boldsymbol{\left\{
                    \frac{0.6}{\mathrm{van}} +
                    \frac{0.3}{\mathrm{motor cycle}} +
                    \frac{0.8}{\mathrm{boat}} +
                    \frac{0.9}{\mathrm{scooter}} +
                    \frac{0.1}{\mathrm{house}}
                    \right\}}$
                \vspace{8pt}

          \item $\overline{\mathrm{Scooter} \cap \mathrm{Scooter}} = \overline{\mathrm{Scooter}}
                  = \boldsymbol{\left\{
                    \frac{0.6}{\mathrm{van}} +
                    \frac{0.3}{\mathrm{motor cycle}} +
                    \frac{0.8}{\mathrm{boat}} +
                    \frac{0.9}{\mathrm{scooter}} +
                    \frac{0.1}{\mathrm{house}}
                    \right\}}$
                \vspace{8pt}

          \item $\mathrm{Scooter} \cup \overline{\mathrm{Van}} = \boldsymbol{\left\{
                    \frac{0.6}{\mathrm{van}} +
                    \frac{0.8}{\mathrm{motor cycle}} +
                    \frac{0.8}{\mathrm{boat}} +
                    \frac{0.9}{\mathrm{scooter}} +
                    \frac{0.8}{\mathrm{house}}
                    \right\}}$
                \vspace{8pt}
          \item $\mathrm{Van} \cup \overline{\mathrm{Van}} = \boldsymbol{\left\{
                    \frac{1.0}{\mathrm{van}} +
                    \frac{0.8}{\mathrm{motor cycle}} +
                    \frac{0.5}{\mathrm{boat}} +
                    \frac{0.7}{\mathrm{scooter}} +
                    \frac{0.8}{\mathrm{house}}
                    \right\}}$
                \vspace{8pt}

          \item $\mathrm{Van} \cap \overline{\mathrm{Van}} = \boldsymbol{\left\{
                    \frac{0.0}{\mathrm{van}} +
                    \frac{0.2}{\mathrm{motor cycle}} +
                    \frac{0.5}{\mathrm{boat}} +
                    \frac{0.3}{\mathrm{scooter}} +
                    \frac{0.2}{\mathrm{house}}
                    \right\}}$
                \vspace{8pt}

        \end{enumerate}

  \item Consider flight simulator data, the determination of certain changes in creating conditions of the aircraft is made on the basis of hard breakpoint in the mach region. Let us define a fuzzy set to represent the condition of near a match number of 0.644. A second fuzzy sets in the region of mach number 0.74.

        \begin{gather*}
          A = \mathrm{near \, mach \,} 0.64
          = \left\{
          \frac{0.1}{0.630} +
          \frac{0.6}{0.635} +
          \frac{1.0}{0.640} +
          \frac{0.8}{0.645} +
          \frac{0.2}{0.650}
          \right\}, \\[12pt]
          B = \mathrm{near \, mach \, 0.64} = \left\{
          \frac{0.0}{0.630} +
          \frac{0.5}{0.635} +
          \frac{0.8}{0.640} +
          \frac{1.0}{0.645} +
          \frac{0.4}{0.650}
          \right\}.
        \end{gather*}

        Find the following:

        \begin{enumerate}
          \item $A \cup B = \boldsymbol{\left\{
                    \frac{0.1}{0.630} +
                    \frac{0.6}{0.635} +
                    \frac{1.0}{0.640} +
                    \frac{1.0}{0.645} +
                    \frac{0.4}{0.650}
                    \right\}}$
                \vspace{8pt}

          \item $A \cap B = \boldsymbol{\left\{
                    \frac{0.0}{0.630} +
                    \frac{0.5}{0.635} +
                    \frac{0.8}{0.640} +
                    \frac{0.8}{0.645} +
                    \frac{0.2}{0.650}
                    \right\}}$
                \vspace{8pt}

          \item $\overline{A} = \boldsymbol{\left\{
                    \frac{0.9}{0.630} +
                    \frac{0.4}{0.635} +
                    \frac{0.0}{0.640} +
                    \frac{0.2}{0.645} +
                    \frac{0.8}{0.650}
                    \right\}}$
                \vspace{8pt}

          \item $\overline{B} = \boldsymbol{\left\{
                    \frac{1.0}{0.630} +
                    \frac{0.5}{0.635} +
                    \frac{0.2}{0.640} +
                    \frac{0.0}{0.645} +
                    \frac{0.6}{0.650}
                    \right\}}$
                \vspace{8pt}

          \item $A \setminus B = A \cap \overline{B}
                  = \boldsymbol{\left\{
                    \frac{0.1}{0.630} +
                    \frac{0.5}{0.635} +
                    \frac{0.2}{0.640} +
                    \frac{0.0}{0.645} +
                    \frac{0.2}{0.650}
                    \right\}}$
                \vspace{8pt}

          \item $\overline{A \cup B} = \overline{A} \cap \overline{B}
                  = \boldsymbol{\left\{
                    \frac{0.9}{0.630} +
                    \frac{0.4}{0.635} +
                    \frac{0.0}{0.640} +
                    \frac{0.0}{0.645} +
                    \frac{0.6}{0.650}
                    \right\}}$
                \vspace{8pt}

          \item $\overline{A \cap B} = \overline{A} \cup \overline{B}
                  = \boldsymbol{\left\{
                    \frac{1.0}{0.630} +
                    \frac{0.5}{0.635} +
                    \frac{0.2}{0.640} +
                    \frac{0.2}{0.645} +
                    \frac{0.8}{0.650}
                    \right\}}$
                \vspace{8pt}
        \end{enumerate}

  \item The continuos form of MOSFET and a transistor are shown in figure below. The discretized membership functions are given by the following equations.

        \begin{gather*}
          \mu_m = \left\{
          \frac{0}{0} +
          \frac{0.4}{2} +
          \frac{0.6}{4} +
          \frac{0.7}{6} +
          \frac{0.8}{8} +
          \frac{0.9}{10}
          \right\}, \\[12pt]
          \mu_T = \left\{
          \frac{0}{0} +
          \frac{0.1}{2} +
          \frac{0.2}{4} +
          \frac{0.3}{6} +
          \frac{0.4}{8} +
          \frac{0.5}{10}
          \right\}
        \end{gather*}

        For these two fuzzy calculate the following:

        \begin{enumerate}
          \item ${\mu}_m \cup {\mu}_T = \boldsymbol{\left\{
                    \frac{0}{0} +
                    \frac{0.4}{2} +
                    \frac{0.6}{4} +
                    \frac{0.7}{6} +
                    \frac{0.8}{8} +
                    \frac{0.9}{10}
                    \right\}}$

          \item ${\mu}_m \cap {\mu}_T = \boldsymbol{\left\{
                    \frac{0}{0} +
                    \frac{0.1}{2} +
                    \frac{0.2}{4} +
                    \frac{0.3}{6} +
                    \frac{0.4}{8} +
                    \frac{0.5}{10}
                    \right\}}$

          \item $\overline{{\mu}}_T = 1 - {\mu}_T = \boldsymbol{\left\{
                    \frac{0}{0} +
                    \frac{0.9}{2} +
                    \frac{0.8}{4} +
                    \frac{0.7}{6} +
                    \frac{0.6}{8} +
                    \frac{0.5}{10}
                    \right\}}$

          \item $\overline{{\mu}_m} = 1 - {\mu}_m = \boldsymbol{\left\{
                    \frac{0}{0} +
                    \frac{0.6}{2} +
                    \frac{0.4}{4} +
                    \frac{0.3}{6} +
                    \frac{0.2}{8} +
                    \frac{0.1}{10}
                    \right\}}$

          \item De Morgan's law

                $\boldsymbol{\overline{{\mu}_m \cup {\mu}_T} =
                    \overline{{\mu}_m} \cap \overline{{\mu}_T} =
                    \left\{
                    \frac{0}{0} +
                    \frac{0.6}{2} +
                    \frac{0.4}{4} +
                    \frac{0.3}{6} +
                    \frac{0.2}{8} +
                    \frac{0.1}{10}
                    \right\}}$

                $\boldsymbol{\overline{{\mu}_m \cap {\mu}_T} =
                    \overline{{\mu}_m} \cup \overline{{\mu}_T} =\left\{
                    \frac{0}{0} +
                    \frac{0.9}{2} +
                    \frac{0.8}{4} +
                    \frac{0.7}{6} +
                    \frac{0.6}{8} +
                    \frac{0.5}{10}
                    \right\}}$
        \end{enumerate}

  \item Samples of new microprocessors IC chip are to be sent to several customers for beta testing. The chips are sorted to meet certain maximum electrical characteristics say frequency, and temperature rating, so that the ''best'' chips are distributed to preferred customer 1. Suppose that each sample chip is screened and all chips are found to have a maximum operating frequency in the range 7-15 MHz at 20ºC. Also the maximum operating temperature range (20ºC $\pm \Delta T$) at 8 MHz is determined. Suppose there are eight sample chips with the following electrical characteristics:

        \begin{table}[H]
          \centering
          \caption{Chip number}
          \begin{tabular}{ccccccccc}
            \toprule
                                      & 1 & 2 & 3  & 4  & 5  & 6  & 7  & 8  \\
            \midrule
            $F_{\mathrm{max}} (MHz)$  & 6 & 7 & 8  & 9  & 10 & 11 & 12 & 13 \\
            $\Delta T_{\mathrm{max}}$ & 0 & 0 & 20 & 40 & 30 & 50 & 40 & 60 \\
            \bottomrule
          \end{tabular}
        \end{table}

        The following fuzzy sets are defined.

        \begin{align*}
          A & = \text{Set of "Fast" chips = chips with } {\mathnormal{f}}_{\mathrm{max}} \geq 12 \, \mathrm{MHz} \\
            & = \left\{
          \frac{0.0}{1} +
          \frac{0.0}{2} +
          \frac{0.1}{3} +
          \frac{0.1}{4} +
          \frac{0.2}{5} +
          \frac{0.8}{6} +
          \frac{1.0}{7} +
          \frac{1.0}{8}
          \right\}                                                                                               \\[12pt]
          B & = \text{Set of "Fast" chips = chips with } {\mathnormal{f}}_{\mathrm{max}} \geq 8 \, \mathrm{MHz}  \\
            & = \left\{
          \frac{0.1}{1} +
          \frac{0.5}{2} +
          \frac{1.0}{3} +
          \frac{1.0}{4} +
          \frac{1.0}{5} +
          \frac{1.0}{6} +
          \frac{1.0}{7} +
          \frac{1.0}{8}
          \right\}                                                                                               \\[12pt]
          C & = \text{Set of "Fast" chips = chips with } {T}_{\mathrm{max}} \geq 10 \, \text{ºC}                 \\
            & = \left\{
          \frac{0.0}{1} +
          \frac{0.0}{2} +
          \frac{1.0}{3} +
          \frac{1.0}{4} +
          \frac{1.0}{5} +
          \frac{1.0}{6} +
          \frac{1.0}{7} +
          \frac{1.0}{8}
          \right\}                                                                                               \\[12pt]
          D & = \text{Set of "Fast" chips = chips with } {T}_{\mathrm{max}} \geq 50 \, \text{ºC}                 \\
            & = \left\{
          \frac{0.0}{1} +
          \frac{0.6}{2} +
          \frac{0.1}{3} +
          \frac{0.2}{4} +
          \frac{0.5}{5} +
          \frac{0.8}{6} +
          \frac{1.0}{7} +
          \frac{1.0}{8}
          \right\}
        \end{align*}

        Using fuzzy set, illustrate various set operations possible.

        $\boldsymbol{A \cup B =
            \left\{
            \frac{0.0}{1} +
            \frac{0.0}{2} +
            \frac{1.0}{3} +
            \frac{1.0}{4} +
            \frac{1.0}{5} +
            \frac{1.0}{6} +
            \frac{1.0}{7} +
            \frac{1.0}{8}
            \right\}}$

        $\boldsymbol{A \cup C =
            \left\{
            \frac{0.0}{1} +
            \frac{0.0}{2} +
            \frac{1.0}{3} +
            \frac{1.0}{4} +
            \frac{1.0}{5} +
            \frac{1.0}{6} +
            \frac{1.0}{7} +
            \frac{1.0}{8}
            \right\}}$

        $\boldsymbol{A \cup D =
            \left\{
            \frac{0.0}{1} +
            \frac{0.6}{2} +
            \frac{0.1}{3} +
            \frac{0.2}{4} +
            \frac{0.5}{5} +
            \frac{0.8}{6} +
            \frac{1.0}{7} +
            \frac{1.0}{8}
            \right\}}$

        $\boldsymbol{B \cup C =
            \left\{
            \frac{0.1}{1} +
            \frac{0.5}{2} +
            \frac{1.0}{3} +
            \frac{1.0}{4} +
            \frac{1.0}{5} +
            \frac{1.0}{6} +
            \frac{1.0}{7} +
            \frac{1.0}{8}
            \right\}}$

        $\boldsymbol{B \cup D =
            \left\{
            \frac{0.1}{1} +
            \frac{0.6}{2} +
            \frac{1.0}{3} +
            \frac{1.0}{4} +
            \frac{1.0}{5} +
            \frac{1.0}{6} +
            \frac{1.0}{7} +
            \frac{1.0}{8}
            \right\}}$

        $\boldsymbol{C \cup D =
            \left\{
            \frac{0.0}{1} +
            \frac{0.6}{2} +
            \frac{1.0}{3} +
            \frac{1.0}{4} +
            \frac{1.0}{5} +
            \frac{1.0}{6} +
            \frac{1.0}{7} +
            \frac{1.0}{8}
            \right\}}$

        $\boldsymbol{A \cap B =
            \left\{
            \frac{0.0}{1} +
            \frac{0.0}{2} +
            \frac{0.1}{3} +
            \frac{0.1}{4} +
            \frac{0.2}{5} +
            \frac{0.8}{6} +
            \frac{1.0}{7} +
            \frac{1.0}{8}
            \right\}}$

        $\boldsymbol{A \cap C =
            \left\{
            \frac{0.0}{1} +
            \frac{0.0}{2} +
            \frac{0.1}{3} +
            \frac{0.1}{4} +
            \frac{0.2}{5} +
            \frac{0.8}{6} +
            \frac{1.0}{7} +
            \frac{1.0}{8}
            \right\}}$

        $\boldsymbol{A \cap D =
            \left\{
            \frac{0.0}{1} +
            \frac{0.0}{2} +
            \frac{0.1}{3} +
            \frac{0.1}{4} +
            \frac{0.2}{5} +
            \frac{0.8}{6} +
            \frac{1.0}{7} +
            \frac{1.0}{8}
            \right\}}$

        $\boldsymbol{B \cap C =
            \left\{
            \frac{0.0}{1} +
            \frac{0.0}{2} +
            \frac{1.0}{3} +
            \frac{1.0}{4} +
            \frac{1.0}{5} +
            \frac{1.0}{6} +
            \frac{1.0}{7} +
            \frac{1.0}{8}
            \right\}}$

        $\boldsymbol{B \cap D =
            \left\{
            \frac{0.0}{1} +
            \frac{0.5}{2} +
            \frac{0.1}{3} +
            \frac{0.2}{4} +
            \frac{0.5}{5} +
            \frac{0.8}{6} +
            \frac{1.0}{7} +
            \frac{1.0}{8}
            \right\}}$

        $\boldsymbol{C \cap D =
            \left\{
            \frac{0.0}{1} +
            \frac{0.6}{2} +
            \frac{0.1}{3} +
            \frac{0.2}{4} +
            \frac{0.5}{5} +
            \frac{0.8}{6} +
            \frac{1.0}{7} +
            \frac{1.0}{8}
            \right\}}$

        $\boldsymbol{\overline{A} =
            \left\{
            \frac{1.0}{1} +
            \frac{1.0}{2} +
            \frac{0.9}{3} +
            \frac{0.9}{4} +
            \frac{0.8}{5} +
            \frac{0.2}{6} +
            \frac{0.0}{7} +
            \frac{0.0}{8}
            \right\}}$

        $\boldsymbol{\overline{B} =
            \left\{
            \frac{0.9}{1} +
            \frac{0.5}{2} +
            \frac{0.0}{3} +
            \frac{0.0}{4} +
            \frac{0.0}{5} +
            \frac{0.0}{6} +
            \frac{0.0}{7} +
            \frac{0.0}{8}
            \right\}}$

        $\boldsymbol{\overline{C} =
            \left\{
            \frac{1.0}{1} +
            \frac{1.0}{2} +
            \frac{0.0}{3} +
            \frac{0.0}{4} +
            \frac{0.0}{5} +
            \frac{0.0}{6} +
            \frac{0.0}{7} +
            \frac{0.0}{8}
            \right\}}$

        $\boldsymbol{\overline{D} =
            \left\{
            \frac{1.0}{1} +
            \frac{0.4}{2} +
            \frac{0.9}{3} +
            \frac{0.8}{4} +
            \frac{0.5}{5} +
            \frac{0.2}{6} +
            \frac{0.0}{7} +
            \frac{0.0}{8}
            \right\}}$


  \item Consider two fuzzy sets A and B as shown in figure below. Write the fuzzy set using membership definition and find the following properties:

        $\boldsymbol{A =
            \left\{
            \frac{0.0}{2} +
            \frac{0.5}{4} +
            \frac{1.0}{6} +
            \frac{0.5}{8} +
            \frac{0.0}{10} +
            \frac{0.0}{12} +
            \frac{0.0}{14} +
            \frac{0.0}{16}
            \right\}}$

        $\boldsymbol{B =
            \left\{
            \frac{0.0}{2} +
            \frac{0.0}{4} +
            \frac{0.0}{6} +
            \frac{0.5}{8} +
            \frac{0.75}{10} +
            \frac{0.75}{12} +
            \frac{0.5}{14} +
            \frac{0.0}{16}
            \right\}}$

        \begin{enumerate}
          \item $A \cup B \boldsymbol{=
                    \left\{
                    \frac{0.0}{2} +
                    \frac{0.5}{4} +
                    \frac{1.0}{6} +
                    \frac{0.5}{8} +
                    \frac{0.75}{10} +
                    \frac{0.75}{12} +
                    \frac{0.5}{14} +
                    \frac{0.0}{16}
                    \right\}}$

          \item $A \cap B \boldsymbol{=
                    \left\{
                    \frac{0.0}{2} +
                    \frac{0.0}{4} +
                    \frac{0.0}{6} +
                    \frac{0.5}{8} +
                    \frac{0.0}{10} +
                    \frac{0.0}{12} +
                    \frac{0.0}{14} +
                    \frac{0.0}{16}
                    \right\}}$

          \item $\overline{A} \boldsymbol{=
                    \left\{
                    \frac{1.0}{2} +
                    \frac{0.5}{4} +
                    \frac{0.0}{6} +
                    \frac{0.5}{8} +
                    \frac{1.0}{10} +
                    \frac{1.0}{12} +
                    \frac{1.0}{14} +
                    \frac{1.0}{16}
                    \right\}}$

          \item $\overline{B} \boldsymbol{=
                    \left\{
                    \frac{1.0}{2} +
                    \frac{1.0}{4} +
                    \frac{1.0}{6} +
                    \frac{0.5}{8} +
                    \frac{0.25}{10} +
                    \frac{0.25}{12} +
                    \frac{0.5}{14} +
                    \frac{1.0}{16}
                    \right\}}$

          \item $A \setminus B \boldsymbol{= A \cap \overline{B} =
                    \left\{
                    \frac{0.0}{2} +
                    \frac{0.5}{4} +
                    \frac{1.0}{6} +
                    \frac{0.5}{8} +
                    \frac{0.0}{10} +
                    \frac{0.0}{12} +
                    \frac{0.0}{14} +
                    \frac{0.0}{16}
                    \right\}}$

          \item $\overline{A \cup B} \boldsymbol{
                    = \overline{A} \cap \overline{B} =
                    \left\{
                    \frac{1.0}{2} +
                    \frac{0.5}{4} +
                    \frac{0.0}{6} +
                    \frac{0.5}{8} +
                    \frac{0.25}{10} +
                    \frac{0.25}{12} +
                    \frac{0.5}{14} +
                    \frac{1.0}{16}
                    \right\}}$
        \end{enumerate}

  \item Consider two fuzzy sets A and B as shown

        \begin{gather*}
          A = \left\{
          \frac{0.0}{1} +
          \frac{0.5}{2} +
          \frac{0.3}{3} +
          \frac{0.7}{4} +
          \frac{0.9}{5}
          \right\}, \\[12pt]
          B = \left\{
          \frac{0.2}{1} +
          \frac{0.4}{2} +
          \frac{0.6}{3} +
          \frac{0.9}{4} +
          \frac{0.4}{5}
          \right\}.
        \end{gather*}

        Find

        \begin{enumerate}
          \item $A \cup B \boldsymbol{=
                    \left\{
                    \frac{0.2}{1} +
                    \frac{0.5}{2} +
                    \frac{0.6}{3} +
                    \frac{0.9}{4} +
                    \frac{0.9}{5}
                    \right\}}$

          \item $A \cap B \boldsymbol{=
                    \left\{
                    \frac{0.0}{1} +
                    \frac{0.4}{2} +
                    \frac{0.3}{3} +
                    \frac{0.7}{4} +
                    \frac{0.4}{5}
                    \right\}}$

          \item $\overline{A} \boldsymbol{=
                    \left\{
                    \frac{1.0}{1} +
                    \frac{0.5}{2} +
                    \frac{0.7}{3} +
                    \frac{0.3}{4} +
                    \frac{0.1}{5}
                    \right\}}$

          \item $\overline{B} \boldsymbol{=
                    \left\{
                    \frac{0.8}{1} +
                    \frac{0.6}{2} +
                    \frac{0.4}{3} +
                    \frac{0.1}{4} +
                    \frac{0.6}{5}
                    \right\}}$

          \item $A \setminus B \boldsymbol{= A \cap \overline{B} =
                    \left\{
                    \frac{0.0}{1} +
                    \frac{0.5}{2} +
                    \frac{0.3}{3} +
                    \frac{0.1}{4} +
                    \frac{0.6}{5}
                    \right\}}$

          \item $\overline{A \cup B} \boldsymbol{
                    = \overline{A} \cap \overline{B} =
                    \left\{
                    \frac{0.8}{1} +
                    \frac{0.5}{2} +
                    \frac{0.4}{3} +
                    \frac{0.1}{4} +
                    \frac{0.1}{5}
                    \right\}}$

        \end{enumerate}

  \item Prove why law of excluded middle and law of contradiction does not hold good for fuzzy.

        \textbf{Ans:}
        In crisp logic, two contradictory statements cannot be true at the same time. Therefore, \(A \cap \overline{A} = \emptyset\). However, in fuzzy logic, the law of non-contradiction does not hold.

        \textit{Proof.} Since
        \[
          \overline{A} = \left\{(x, {\mu}_{\overline{A}}(x)) \mid {\mu}_{\overline{A}}(x) = 1 - {\mu}_{A}(x), \forall x \in U\right\}
        \]
        and
        \[
          A \cap B = \left\{(x, {\mu}_{{A \cap B}}(x)) \mid {\mu}_{{A \cap B}}(x) = \min\left({\mu}_{A}(x), {\mu}_{B}(x)\right), \forall x \in U\right\}
        \]
        then,
        \begin{align*}
          A \cap \overline{A} & = \left\{(x, {\mu}_{{A \cap \overline{A}}}(x)) \mid {\mu}_{{A \cap \overline{A}}}(x) = \min\left({\mu}_{A}(x), {\mu}_{\overline{A}}(x)\right), \forall x \in U\right\} \\
          A \cap \overline{A} & = \left\{(x, {\mu}_{{A \cap \overline{A}}}(x)) \mid {\mu}_{{A \cap \overline{A}}}(x) = \min\left({\mu}_{A}(x), 1 - {\mu}_{A}(x)\right), \forall x \in U\right\}
        \end{align*}
        Therefore, if \(A \neq \emptyset\), then \(A \cap \overline{A} \neq \emptyset\), such that \({\mu}_{{A \cap \overline{A}}}(x) \in (0, 0.5)\), and thus, there will always be at least one singleton \(\frac{{\mu}_{{A \cap \overline{A}}}(x)}{x}\).

        The law of excluded middle in crisp logic states that a proposition is true or its negation is true, that is, \(A \cup \overline{A} = U\). However, in fuzzy logic, this is not true.

        \textit{Proof.} Using De Morgan's law, we have:

        \[
          A \cup \overline{A} = \overline{\overline{A} \cap A} = \overline{A \cap \overline{A}}
        \]

        thus, \({\mu}_{A \cup \overline{A}} = 1 - {\mu}_{{A \cap \overline{A}}}(x)\), and therefore, \({\mu}_{A \cup \overline{A}} \in (0.5, 1)\). Thus, \(A \cup \overline{A} = U\) will not always be true, as long as at least one \({\mu}_{A} \in (0, 0.5)\) in a given universe of discourse, the proposition becomes false.

  \item Consider the universe with two elements $X = \left\{a, b\right\}$ and consider $Y$ with $Y = \left\{0, 1\right\}$. Find the power set.

        \textbf{Ans:}
        $
          \mathcal{P}(X) = \left\{\emptyset, \{a\}, \{b\}, \{a, b\}\right\} \, e \, \mathcal{P}(Y) = \left\{\emptyset, \{0\}, \{1\}, \{0, 1\}\right\}
        $

  \item Consider a universe of four elements $x = \left\{1, 2, 3, 4, 5, 6\right\}$. Find the cardinal number power set and cardinality.

        \textbf{Ans:}
        $
          n_{x} = 6 \, \text{ e } \, \eta_{\mathcal{P}(x)} = 2 ^ {n_x} = 2 ^ 6 = 64
        $

  \item Consider the following fuzzy sets:
        \begin{gather*}
          A = \left\{
          \frac{1.0}{2} +
          \frac{0.1}{3} +
          \frac{0.8}{4} +
          \frac{0.6}{5}
          \right\}, \\[12pt]
          B = \left\{
          \frac{0.3}{2} +
          \frac{0.9}{3} +
          \frac{0.0}{4} +
          \frac{0.4}{5}
          \right\}.
        \end{gather*}

        Calculate, $A \cup B$, $A \cap B$, $\overline{A}$, $\overline{B}$ by Matlab program.

        \textbf{Ans:}

        $
          A \cup B \boldsymbol{=
            \left\{
            \frac{1.0}{2} +
            \frac{0.9}{3} +
            \frac{0.8}{4} +
            \frac{0.6}{5}
            \right\}}
        $

        $
          A \cap B \boldsymbol{=
            \left\{
            \frac{0.3}{2} +
            \frac{0.1}{3} +
            \frac{0.0}{4} +
            \frac{0.4}{5}
            \right\}}
        $

        $
          \overline{A}  \boldsymbol{=
            \left\{
            \frac{0.0}{2} +
            \frac{0.9}{3} +
            \frac{0.2}{4} +
            \frac{0.4}{5}
            \right\}}
        $

        $
          \overline{B}  \boldsymbol{=
            \left\{
            \frac{0.7}{2} +
            \frac{0.1}{3} +
            \frac{1.0}{4} +
            \frac{0.6}{5}
            \right\}}
        $

  \item For the above problem perform the De Morgan's law by writing M-file.

        \textbf{Ans:}

        $
          \overline{A \cup B} \boldsymbol{=
            \left\{
            \frac{0.0}{2} +
            \frac{0.1}{3} +
            \frac{0.2}{4} +
            \frac{0.4}{5}
            \right\}}
        $

        $
          \overline{A \cap B} \boldsymbol{=
            \left\{
            \frac{0.7}{2} +
            \frac{0.9}{3} +
            \frac{1.0}{4} +
            \frac{0.6}{5}
            \right\}}
        $

\end{enumerate}

\end{document}
